\section{The mDNS protocol}
The Multicast DNS (mDNS) protocol is a decentralized and self-organizing network service discovery mechanism. It allows devices on a local network to discover and communicate with each other without the need for a centralized DNS server. Unlike traditional DNS, which relies on centralized servers to resolve domain names to IP addresses, mDNS operates within a single local network and enables devices to discover each other's services and addresses.

mDNS utilizes multicast packets to communicate. Multicast is a form of communication where a single packet is sent to multiple recipients simultaneously. In the case of mDNS, devices interested in service discovery listen to a specific multicast address (IPv4: 224.0.0.251, IPv6: FF02::FB) and respond accordingly.

\subsection{Packets structure}
In the mDNS protocol, communication is done using multicast packets. These packets have a specific structure, consisting of header fields and data sections. The header contains information such as the source and destination IP addresses, port numbers, and protocol version. The data sections carry the actual content of the mDNS messages.
\subsection{Queries}
mDNS queries are used to discover services and devices on the local network. A query packet is sent to the multicast address, and devices that provide the requested service respond with a unicast packet containing the necessary information. Queries can be specific to a particular service or can be a general request for all available services.

\begin{table}[H]
    \centering
    \caption{\textbf{mDNS Query section fields}}
    \begin{tabularx}{\linewidth}{|X|p{0.4\linewidth}|X|}
    \hline
    \textbf{Field} & \textbf{Description} & \textbf{Length bits}\\ \hline
    QNAME & Name of the node to which the query pertains & Variable   \\ \hline
    QTYPE & The type of the query, i.e. the type of RR which should be returned in responses. & 16\\ \hline
    UNICAST-RESPONSE & Boolean flag indicating whether a unicast-response is desired & 1\\ \hline
    QCLASS & Class code, 1 a.k.a. "IN" for the Internet and IP networks & 15\\ \hline
    \end{tabularx}
    \label{tab:query}
\end{table}


\subsection{Resource Records}
Resource records in mDNS provide information about services, devices, and other network resources. They contain data such as IP addresses, service names, port numbers, and other attributes. Resource records are used in both query and response packets to facilitate service discovery and communication. \\
There are different types of resource records, including:
\begin{itemize}
    \item Address (A) Record: Maps a hostname to an IPv4 address.
    \item AAAA Record: Maps a hostname to an IPv6 address.
    \item Service (SRV) Record: Provides information about a particular service, such as the port number and the hostname offering the service.
    \item Pointer (PTR) Record: Associates a name with a resource record, typically used for service discovery.
    \item Text (TXT) Record: Carries additional information about a service, such as descriptive text or configuration parameters.
\end{itemize}

\begin{table}[H]
    \centering
    \caption{\textbf{mDNS Resource Record fields}}
    \begin{tabularx}{\linewidth}{|X|p{0.4\linewidth}|X|}
    \hline
    \textbf{Field} & \textbf{Description} & \textbf{Length bits}\\ \hline
    RRNAME & Name of the node to which the record pertains & Variable   \\ \hline
    RRTYPE & The type of the Resource Record & 16\\ \hline
    CACHE-FLUSH & Boolean flag indicating whether outdated cached records should be purged & 1\\ \hline
    RRCLASS & Class code, 1 a.k.a. "IN" for the Internet and IP networks & 15\\ \hline
    TTL & Time interval (in seconds) that the RR should be cached & 32\\ \hline
    RDLENGTH & Integer representing the length (in octets) of the RDATA field & 16\\ \hline
    RDATA & Resource data; internal structure varies by RRTYPE & Variable\\ \hline
    \end{tabularx}
    \label{tab:resource}
\end{table}


