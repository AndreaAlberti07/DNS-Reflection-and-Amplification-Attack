\section{Methodologies and experimental setup}

To ensure the success of the project, it is essential to establish a clear methodology and employ a well-defined set of tools.
This section aims to provide a detailed explanation of the methodologies utilized to accomplish the project's objectives.\\
The selected methodologies encompass a systematic approach that allows for the accurate replication of a
DDoS attack while maintaining ethical considerations and minimizing the potential impact on live networks.
These methodologies were carefully chosen to ensure the reliability and validity of the experimental results.
The methodological approach used is the following:\\
\textbf{Why}\\
The objective of this study is to assess the impact of a DoS attack, exploiting the DNS protocol and monitor the reachability
of the targeted device and other network nodes.
By simulating realistic attack scenarios, this study aims to understand the vulnerabilities within the DNS protocol, evaluate network resilience,
and identify potential countermeasures.\\
\textbf{Which/Who}\\
The chosen target for the DDoS attack is a laptop (HP ENVY x360), which acts as a host for a DNS server running BIND9.
Two laptops, specifically a MacBook Pro 14 and another device (to be specified)!!!!!!!!!!!!!!!!, are utilized as vantage points to initiate the attack.\\
\textbf{What}\\
The chosen metrics include measuring the response time for each ICMP or DNS message sent, accounting for potential timeouts.
Furthermore, the CPU and memory utilization of the DNS server were also monitored during the simulation. \\
\textbf{Where}\\
The vantage point is a MacBook Air inside the
network and passive to the attack.\\
To ensure the security and stability of public networks and devices,
the DDoS attack simulations were conducted within a local area network (LAN) environment.
