\section{DNS based DDoS attacks}
The DNS service is a critical part of the Internet infrastructure. Almost every communication exploits the DNS service, and attacking
it could lead a significant number of networked applications to run out of service. The DNS service has been designed to provide fast 
responses and is less oriented towards security, which makes it vulnerable to different types of attacks. According to a recent Cloudflare's
report \cite{DDoSthreatreport} the most used vector in DDoS is the DNS. The types of attacks involving DNS are of three main types: 
\textit{DNS query flood}, \textit{TCP flood}, \textit{DNS reflection and amplification}, and \textit{DNS water torture}.\\
\\
\textbf{DNS query flood}\\
This is a direct attack aimed at consuming the server resources until run it out of service. The attacker sends a large number of DNS queries
to the target recursive DNS server, leveraging a zombie devices army (botnet). The requests are structured in a way that the server has not the
record cached and it is forced to perform recursive queries to provide the responses. It is easier to perform this type of attack on smaller 
DNS servers that may have limited resources to handle a large number of queries.\\
\\
\textbf{TCP flood}\\
This is another type of attack aimed at consuming the server resources. These latter are consumed by the attacker sending lots of TCP 
connection requests without closing them. The server is forced to allocate resources to handle the TCP connections and, when the 
numbers get larger it may run out of resources.\\
\\
\textbf{DNS water torture}\\
This is a type of indirect attack aimed at consuming the resources of a target authoritative DNS server. The attacker sends a huge number
of queries about properly constructed hostnames. These are made up by two parts: the domain whose authoritative server is the target, and
a random string such that the FQDN cannot exist. That way the recursive NS starts its search until it reaches the authoritative server. This latter
notices the non-existence of the FQDN and sends NXDOMAIN response. All the queries travel until the authoritative server, which is forced to
handle all of them until it is overwhelmed.\\
\\
\textbf{DNS reflection and amplification}\\
This is an indirect attack aimed at consuming the target network bandwidth. It start with the spoofing of the target's IP address, then lots 
of queries are sent to the DNS server using as the source address the spoofed IP address. This latter will receives the responses from the 
DNS server (reflection). The queries are structured in a way that the responses are larger in size (amplification). That way the target's
bandwidth is fulfilled, with the attacker having used just a small quantity of its resources. The effectiveness of this attack is given by 
the amplification factor (AF), measured as the ratio between the size of the response and the size of the query. An easy but effective choice
for the attacker is to perform a type ANY which provides a large AF.\\
\\
Since according to the already cited report \cite{DDoSthreatreport} the most common type of attack is the DNS reflection and amplification,
in this project we focus on this type of attack.


