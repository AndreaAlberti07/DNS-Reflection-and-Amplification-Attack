\section{DNS based DDoS attacks}
The DNS service plays a crucial role in the Internet infrastructure as it is relied upon by almost every internet service. 
Any compromise to the DNS service could have significant consequences, leading to a disruption of numerous networked applications. 
While the primary focus of DNS design is to provide fast responses, its emphasis on security is relatively limited, making 
it susceptible to various forms of attacks. According to a recent report by Cloudflare (\cite{DDoSthreatreport}), almost a third
of all DDoS attacks are DNS-based. These attacks primarily fall into three categories:
\textit{DNS query flood}, \textit{TCP flood}, and \textit{DNS reflection and amplification}.\\
\\
\textbf{DNS query flood}\\
This is a direct attack, conducted with the intention of overwhelming the target DNS server by exhausting its available resources, 
eventually causing it to become unresponsive. This type of attack involves the attacker leveraging a botnet, a network of 
compromised devices, to send a massive volume of DNS queries to the target server.

When the victim of the attack is a recursive DNS server, the requests are structured in such a way that the server doesn't 
have the requested records cached. As a result, the server is forced to perform recursive queries to obtain the requested 
information and provide responses to the attacker, using eventually all its available resources.

A variant of this attack, known as \textbf{DNS water torture} attack, specifically targets authoritative DNS servers. 
In this scenario, the attacker floods the server with an enormous number of properly constructed queries.
These latter consist of two parts: the domain of the victim authoritative server and a random 
string that ensures the fully qualified domain name (FQDN) does not exist. As a result, the recursive nameserver initiates 
a search that eventually reaches the targeted authoritative server. The authoritative server, upon realizing that the FQDN 
does not exist, sends an NXDOMAIN response. All of these queries travel to the authoritative server, overwhelming its 
resources as it is forced handling each request.

The DNS query flood attack is particularly effective against smaller DNS servers that may have limited resources to handle 
the high volume of queries, making them more vulnerable to disruption.\\
\\
\textbf{TCP flood}\\
Another form of attack that aims to exhaust server resources is the TCP flood attack. In this type of attack, the attacker inundates the 
target server with a large number of TCP connection requests but does not close these connections. As a result, the server is compelled 
to allocate resources to handle each incoming TCP connection. As the number of connections grows, if the attack is successful the server 
becomes unresponsive to legitimate users.\\
\\
\textbf{DNS reflection and amplification}\\
DNS reflection and amplification is an indirect attack strategy that aims to consume the target network's bandwidth. The attack 
begins with the attacker spoofing the IP address of the target. Subsequently, numerous queries are sent to a DNS server, utilizing the 
spoofed IP address as the source address. In this scenario, the spoofed IP address receives the responses from the DNS server, which 
is the \textit{reflection} aspect of the attack.
To amplify the impact, the queries are crafted in a way that the responses from the DNS server are significantly larger 
in size. This \textit{amplification} aspect ensures that the target's bandwidth becomes overwhelmed, while the attacker only expends minimal resources.
The effectiveness of this attack is determined by the Amplification Factor (AF), which is calculated by comparing the size of the response to 
the size of the query. To achieve a high amplification factor, attackers often perform a type ANY query, which provides a 
substantial amplification effect.
\\
Since according to the already cited report (\cite{DDoSthreatreport}) the most common type of attack is the DNS reflection and amplification,
in this project we focus on this type of attack.


