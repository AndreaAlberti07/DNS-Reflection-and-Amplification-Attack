\section{Mitigation measures}
There exist a variety of measures that can be employed to mitigate the effects of DNS amplification attacks. 
These measures can be broadly categorized into two groups: those that aim to reduce the probability of an attack occurring, and 
those that aim to minimize the impact of an attack by detecting it early and enhancing the resilience of the DNS service.

\subsection{Proactive measures}
\textbf{Rate limiting}\\
Rate limiting is a measure that can be used to mitigate the impact of DNS amplification attacks. The idea behind rate limiting is to limit 
the number of responses that a DNS server can send to a specific IP address within a certain time period. That way the queries sent
by the attacker are dropped by the DNS server, thus reducing the amplification effect.\\
\\
\textbf{Trusted sources}\\
When a DNS recursive server is open on the Internet, it can receive queries from any source. The range of IP addresses that can 
be spoofed is very large and it is not possible to block all of them. However, it is possible to limit the number of sources that can
send queries to the DNS server, creating a whitelist of trusted sources. This measure reduce the probability of an attack occurring, however,
the trusted sources could be spoofed, thus the attack be performed.\\
\\
\textbf{Firewall}\\
A firewall is a network security system that monitors and controls the incoming and outgoing network traffic based on predetermined security
rules. Setting up a proper firewall both DNS server side and victim side can block unauthorized traffic and reduce the impact of the attack.

\subsection{Detection measures}
The DNS amplification attack has as core activity the IP spoofing. A mechanism able to discriminate between a original source IP address
and a spoofed could detect the attack. This is the main idea behind the \textit{detection} techniques.\\
\\
\textbf{Routing hops detection}\\
This mechanism was proposed by \citeauthor{hopcount} in this paper \cite{hopcount}. The idea is to exploit the inconsistency between the
number of hops of a spoofed IP packet and the spoofed IP address itself. The hops number is inferred by the TTL value in the IP header.
This mechanism can detect almost 90\% of the spoofed packets.\\
\\
\textbf{Machine Learning}\\
In the last decade with the developing of machine learning, some algorithms have been proposed to detect the DNS amplification attack.
In 2015 \citeauthor{machinelearning} \cite{machinelearning} proposed a machine learning based approach to detect the attacks, using 
\textit{Random Forest}, \textit{MLP} and \textit{SVM} algorithms. However, a more recent publication \cite{createDNS} by 
\citeauthor{createDNS}, shows that using an adversarial neural network approach (EAD) it is possible to easy circumvent the detection. 
The idea is to train a network to slightly modify the input data (DNS queries) in order to fool the detection algorithm.

\subsection{Resilience measures}
These measures are more focused on making the DNS more robust to the attack, allowing it to continue to provide the service 
even during the attack.\\
\\
\textbf{Anycast scheme}\\
The DDoS attack is aimed at causing a service outage on a victim server by flooding it with a large number of packets. The idea behind the 
anycast solution is to have many replicas of the victim server with same logical IP address, and to choose the destination of the packets
according to some routing criteria. That way, the packets are distributed among the replicas, thus reducing the load on each of them.
In 2015, a large-scale DDoS attack was launched against the DNS root name servers, resulting in denial of service for some of them, as reported by 
\cite{anycast}. The attack demonstrated that although the use of anycast schema can increase the resilience of DNS servers, it cannot entirely 
protect against DDoS attacks. However, due to the deployment of anycast technology in 11 out of 13 root servers, the impact of the attack was 
limited and partially mitigated.\\
\\
\textbf{Caching behavior}\\
This approach is discussed by \citeauthor{alleviatingimpact} in this paper \cite{alleviatingimpact}. They support that a relatively 
simple change in the caching behavior of a DNS server can significantly improve the DNS performance under a DDoS attack.
They propose a DNS server should not evict the cache entries when it detects the relevant DNS servers are unavailable and delete
them as soon as this latter become available again. That way, even during an attack running out of service a relevant server, the 
DNS recursive server can still serve the cached entries, thus providing part of the requested domain names.\\


