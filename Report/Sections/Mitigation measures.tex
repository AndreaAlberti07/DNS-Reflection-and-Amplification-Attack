\section{Mitigation measures}
There exist a variety of measures that can be employed to mitigate the effects of DNS amplification attacks. 
These measures can be broadly categorized into three groups: those that aim to reduce the probability of an attack occurring, 
those that aim to minimize the impact of an attack by detecting it early, and those oriented to enhancing the resilience of the DNS service.

\subsection{Proactive measures}
\textbf{Rate limiting}\\
Rate limiting is a measure that can be used to mitigate the impact of DNS amplification attacks. The idea behind rate limiting is to limit 
the number of responses that a DNS server can send to a specific IP address within a certain time period. That way the queries sent
by the attacker are dropped by the DNS server, therefore reducing the amplification effect.\\
\\
\textbf{Trusted sources}\\
When a DNS recursive server is publicly accessible on the Internet, it becomes susceptible to receiving queries from any source. 
The potential range of IP addresses that can be spoofed is vast, making it impractical to block all of them effectively. However, a 
possible mitigation strategy is to restrict the number of sources allowed to send queries to the DNS server by implementing a whitelist of trusted 
sources. By creating such a whitelist, the probability of an attack occurring can be reduced. It is worth nothing to underline that 
the introduction of a whitelist is a not a foolproof method, since IP in whitelist can still be spoofed and used to perform the attack.\\
\\
\textbf{Firewall}\\
A firewall is a network security system that monitors and controls the incoming and outgoing network traffic based on predetermined security
rules. Setting up a proper firewall both DNS server side and victim side can block unauthorized traffic and reduce the impact of the attack.

\subsection{Detection measures}
The DNS amplification attack relies heavily on IP spoofing as a central activity. The key concept behind \textit{detection} techniques is the ability 
to differentiate between an original source IP address and a spoofed one, therefore immediately identify and mitigate the attack. \\
\\
\textbf{Routing hops detection}\\
This mechanism was proposed in the paper by '\cite{hopcount}'. The idea is to exploit the inconsistency between the
number of hops of a spoofed IP packet and the spoofed IP address itself. The hops number is inferred by the TTL value in the IP header.
This mechanism can detect almost 90\% of the spoofed packets.\\
\\
\textbf{Machine Learning}\\
In the last decade with the developing of machine learning, some algorithms have been proposed to detect the DNS amplification attack.
In the paper by '\cite{machinelearning}', is proposed a machine learning based approach to detect the attacks, using 
\textit{Random Forest}, \textit{MLP} and \textit{SVM} algorithms. However, a more recent publication by 
'\cite{createDNS}', shows that using an adversarial neural network approach (EAD) it is possible to easy circumvent the detection. 
The idea is to train a network to slightly modify the input data (DNS queries) in order to fool the detection algorithm.

\subsection{Resilience measures}
Resilience measures aim to enhance the DNS's ability to withstand attacks and maintain its service even when under attack.\\
\\
\textbf{Anycast scheme}\\
A DDoS attack aims to disrupt a victim server's service by overwhelming it with a high volume of packets. A solution to this issue involves 
creating multiple replicas of the victim server, all sharing the same logical IP address. The destination of 
incoming packets is determined by the Anycast protocol employing certain routing criteria. Thereby, the load is distributed among the replicas 
and the strain on each individual server is reduced.
In 2015, a significant DDoS attack targeted the DNS root name servers, resulting in a denial of service for some of them, as documented in the report '\cite{anycast}'. 
This attack highlighted that although anycast architecture can enhance the resilience of DNS servers, it is not foolproof solution to DDoS attacks. 
Nevertheless, due to the implementation of anycast technology in 11 out of the 13 root servers, the impact of the attack was limited and partially mitigated.\\
\\
\textbf{Caching behavior}\\
The paper authored by '\cite{alleviatingimpact}' explores an approach to mitigate the impact of DDoS attacks on DNS performance. 
The authors argue that a relatively simple modification to the caching behavior of a DNS server can yield significant 
improvements in performance during such attacks.
Their proposal suggests that a DNS server should refrain from evicting cache entries when it detects unavailability of the relevant DNS servers. 
Instead, these entries should be retained until the corresponding servers become available again. By implementing this change, even if a relevant 
server is rendered inaccessible during a DDoS attack, the DNS recursive server can still serve the cached entries, thereby providing responses for 
a portion of the requested domain names.


