\documentclass{class}
\usepackage{multicol}
\title{EDI project: DNS amplification attack}

\author{Andreoli C. • Ligari D. • Alberti A. • Scardovi M. • Intini K.}



\affil[1]{Department of Computer Engineering - Data Science, University of Pavia, Italy}
%% Corresponding author
\corrauthor{Author One}

%% Abbreviated author list for the running footer
\runningauthor{Alberti, Andreoli, Intini, Ligari, Scardovi}
\shortname{DNS amplification attack}
\bibliography{References}

\begin{document}

\maketitle
\begin{abstract}
    Only two levels of sectional headings, \verb|\section| and \verb|\subsection|, should be used. Ad nemo aut quae dolores nesciunt reprehenderit occaecati. Optio distinctio at aliquam odit dolores laudantium. Illum et qui iste et laudantium dolorum. Nihil quis qui at quia alias. Quisquam ea sit aspernatur. Labore at hic voluptas cumque eum officia repellat.

    \keywords{DNS server • DDoS Attack • Wireshark • Dig • Top • Ping}
\end{abstract}
\begin{multicols}{2}
    \cite{dizdar_dns_2021}
    \cite{dnsbaseddos}
    \cite{alleviatingimpact}
    \cite{taylor_four_2021}
    \cite{DNSattackstype}
    \cite{DDoSthreatreport}
    \cite{Devi_2015}
    \cite{anycast}
    \cite{machinelearning}
    \cite{createDNS}
    %Andrea
    \section{DNS based DDoS attacks}
The DNS service plays a crucial role in the Internet infrastructure as it is relied upon by almost every internet service. 
Any compromise to the DNS service could have significant consequences, leading to a disruption of numerous networked applications. 
While the primary focus of DNS design is to provide fast responses, its emphasis on security is relatively limited, making 
it susceptible to various forms of attacks. According to a recent report by Cloudflare (\cite{DDoSthreatreport}), almost a third
of all DDoS attacks are DNS-based. These attacks primarily fall into three categories:
\textit{DNS query flood}, \textit{TCP flood}, and \textit{DNS reflection and amplification}.\\
\\
\textbf{DNS query flood}\\
This is a direct attack, conducted with the intention of overwhelming the target DNS server by exhausting its available resources, 
eventually causing it to become unresponsive. This type of attack involves the attacker leveraging a botnet, a network of 
compromised devices, to send a massive volume of DNS queries to the target server.

When the victim of the attack is a recursive DNS server, the requests are structured in such a way that the server doesn't 
have the requested records cached. As a result, the server is forced to perform recursive queries to obtain the requested 
information and provide responses to the attacker, using eventually all its available resources.

A variant of this attack, known as \textbf{DNS water torture} attack, specifically targets authoritative DNS servers. 
In this scenario, the attacker floods the server with an enormous number of properly constructed queries.
These latter consist of two parts: the domain of the victim authoritative server and a random 
string that ensures the fully qualified domain name (FQDN) does not exist. As a result, the recursive nameserver initiates 
a search that eventually reaches the targeted authoritative server. The authoritative server, upon realizing that the FQDN 
does not exist, sends an NXDOMAIN response. All of these queries travel to the authoritative server, overwhelming its 
resources as it is forced handling each request.

The DNS query flood attack is particularly effective against smaller DNS servers that may have limited resources to handle 
the high volume of queries, making them more vulnerable to disruption.\\
\\
\textbf{TCP flood}\\
Another form of attack that aims to exhaust server resources is the TCP flood attack. In this type of attack, the attacker inundates the 
target server with a large number of TCP connection requests but does not close these connections. As a result, the server is compelled 
to allocate resources to handle each incoming TCP connection. As the number of connections grows, if the attack is successful the server 
becomes unresponsive to legitimate users.\\
\\
\textbf{DNS reflection and amplification}\\
DNS reflection and amplification is an indirect attack strategy that aims to consume the target network's bandwidth. The attack 
begins with the attacker spoofing the IP address of the target. Subsequently, numerous queries are sent to a DNS server, utilizing the 
spoofed IP address as the source address. In this scenario, the spoofed IP address receives the responses from the DNS server, which 
is the \textit{reflection} aspect of the attack.
To amplify the impact, the queries are crafted in a way that the responses from the DNS server are significantly larger 
in size. This \textit{amplification} aspect ensures that the target's bandwidth becomes overwhelmed, while the attacker only expends minimal resources.
The effectiveness of this attack is determined by the Amplification Factor (AF), which is calculated by comparing the size of the response to 
the size of the query. To achieve a high amplification factor, attackers often perform a type ANY query, which provides a 
substantial amplification effect.
\\
Since according to the already cited report (\cite{DDoSthreatreport}) the most common type of attack is the DNS reflection and amplification,
in this project we focus on this type of attack.




    %\input{./Sections/Mitigation measures_gitignore.tex}
    %Davide
    \section{Methodologies and experimental setup}

To ensure the success of the project, it is essential to establish a clear methodology and employ a well-defined set of tools.
This section aims to provide a detailed explanation of the methodologies utilized to accomplish the project's objectives.\\
The selected methodologies encompass a systematic approach that allows for the accurate replication of a
DDoS attack while maintaining ethical considerations and minimizing the potential impact on live networks.
These methodologies were carefully chosen to ensure the reliability and validity of the experimental results.
The methodological approach used is the following:\\
\textbf{Why}\\
The objective of this study is to assess the impact of a DoS attack, exploiting the DNS protocol and monitor the reachability
of the targeted device and other network nodes.
By simulating realistic attack scenarios, this study aims to understand the vulnerabilities within the DNS protocol, evaluate network resilience,
and identify potential countermeasures.\\
\textbf{Which/Who}\\
The chosen target for the DDoS attack is a laptop (HP ENVY x360), which acts as a host for a DNS server running BIND9.
Two laptops, specifically a MacBook Pro 14 and another device (to be specified)!!!!!!!!!!!!!!!!, are utilized as vantage points to initiate the attack.\\
\textbf{What}\\
The chosen metrics include measuring the response time for each ICMP or DNS message sent, accounting for potential timeouts.
Furthermore, the CPU and memory utilization of the DNS server were also monitored during the simulation. \\
\textbf{Where}\\
The vantage point is a MacBook Air inside the
network and passive to the attack.\\
To ensure the security and stability of public networks and devices,
the DDoS attack simulations were conducted within a local area network (LAN) environment.

    \section{Tools used}
\textbf{Ping} \\
Ping is a network utility tool used to test the connectivity between two networked devices.
It sends a small packet of data to a specific IP address or hostname and measures the time it takes for that packet to be received and returned.
The result shows the round trip time (RTT), as well as the number of packets sent and received, and any packet loss that may have occurred.\\
While it is a useful tool for testing network connectivity, it has some limitations,
it uses the ICMP protocol to send and receive packets, which may not always be allowed by network firewalls or routers, and it does not support different protocols like TCP or UDP.
This means that if a network is configured to block ICMP traffic, the ping command may not work.
Moreover ping provides only basic information about network connectivity, it does not provide information about bandwidth or the structure of the network.\\
\\
\textbf{Dig} \\
The dig command (short for "domain information groper") is a popular network administration tool used to perform DNS queries.
It allows users to perform DNS lookups to check DNS records and obtain information about DNS configurations.
It is a versatile tool that allows users to specify different query types, such as A, AAAA, MX, TXT, and others.
It can also be used to perform reverse DNS queries, where an IP address is used to retrieve the corresponding domain name.
In addition to provide detailed DNS query results, the dig command can also be used to troubleshoot DNS issues, such as misconfigured DNS servers,
slow DNS resolution times, or DNS cache issues.\\
\\
\textbf{Top} \\
The top command provides real-time monitoring of system resources, such as CPU usage, memory utilization, running processes, and more.
When executed, the top command displays an interactive, dynamic table that continually updates, allowing users to view
the current state of their system and identify any processes consuming excessive resources.
The table is typically sorted by CPU usage by default, but users can customize the sorting order based on their preferences.
The top command also provides options to manipulate the displayed information and perform actions on running processes,
such as terminating or changing their priority.\\
\\
\textbf{Wireshark}\\
Wireshark is a open-source network protocol analyzer.
It is designed to capture, analyze, and display network traffic in real-time.
Wireshark allows users to inspect and interpret the data packets transmitted over a network, providing detailed information
about the communication between different devices.\\
In addition to passive packet capturing, Wireshark offers powerful filtering and search capabilities,
allowing users to focus on specific types of traffic or specific packets of interest.
It also provides features for advanced analysis, such as the ability to reconstruct and view streams of data,
perform statistical analysis, and even export captured data for further investigation or reporting
    \section{DNS Server}
For the simulation of the DDoS attack, a dedicated DNS server was established on a laptop using Ubuntu 24.04 LTS operating system.
The laptop is an HP ENVY x360, with 12 GB of RAM and 4 core.\\
To create the DNS server, the open-source software BIND9 was employed. This widely adopted DNS server software offers robust functionality and configurability.
The server was specifically configured to act as the authoritative server for the domain name \textit{ediproject.com.}
By assuming authority over this domain, the DNS server can respond to DNS queries and provide legitimate DNS responses during the simulation.\\
To ease the attack simulation, the DNS server was configured to respond to all DNS queries with the same LAN, and no security measures were implemented.
To mimic the characteristics of a genuine DNS server, a total of seven NS records, five MX-type records and ten A-type records were added to the DNS server's configuration.
The inclusion of these records introduces diverse amplification factors for different type of records, enabling the analysis of the DDoS attack's effectiveness based on these factors.



    %\input{./Sections/3_section.tex}
    %\input{./Sections/4_section.tex}
    %\input{./Sections/5_conclusion.tex}
    \newpage
    \printbibliography
\end{multicols}
\end{document}
