\documentclass{class}
\usepackage{multicol}
\title{EDI project: DNS amplification attack}

\author{Andreoli C. • Ligari D. • Alberti A. • Scardovi M. • Intini K.}



\affil[1]{Department of Computer Engineering - Data Science, University of Pavia, Italy}
%% Corresponding author
\corrauthor{Author One}

%% Abbreviated author list for the running footer
\runningauthor{Alberti, Andreoli, Intini, Ligari, Scardovi}
\shortname{DNS amplification attack}
\bibliography{References}

\begin{document}

\maketitle
\begin{abstract}
    Only two levels of sectional headings, \verb|\section| and \verb|\subsection|, should be used. Ad nemo aut quae dolores nesciunt reprehenderit occaecati. Optio distinctio at aliquam odit dolores laudantium. Illum et qui iste et laudantium dolorum. Nihil quis qui at quia alias. Quisquam ea sit aspernatur. Labore at hic voluptas cumque eum officia repellat.

    \keywords{DNS server • DDoS Attack • Wireshark • Dig • Top • Ping}
\end{abstract}
\begin{multicols}{2}
    \cite{dizdar_dns_2021}
    \cite{dnsbaseddos}
    \cite{alleviatingimpact}
    \cite{taylor_four_2021}
    \cite{DNSattackstype}
    \cite{DDoSthreatreport}
    \cite{Devi_2015}
    \cite{anycast}
    \cite{machinelearning}
    \cite{createDNS}
    %Andrea
    \section{DNS based DDoS attacks}
The DNS service is a critical part of the Internet infrastructure. Almost every communication exploits the DNS service, and attacking
it could lead a significant number of networked applications to run out of service. The DNS service has been designed to provide fast 
responses and is less oriented towards security, which makes it vulnerable to different types of attacks. According to a recent Cloudflare's
report \cite{DDoSthreatreport} the most used vector in DDoS is the DNS. The types of attacks involving DNS are of three main types: 
\textit{DNS query flood}, \textit{TCP flood}, \textit{DNS reflection and amplification}, and \textit{DNS water torture}.\\
\\
\textbf{DNS query flood}\\
This is a direct attack aimed at consuming the server resources until run it out of service. The attacker sends a large number of DNS queries
to the target recursive DNS server, leveraging a zombie devices army (botnet). The requests are structured in a way that the server has not the
record cached and it is forced to perform recursive queries to provide the responses. It is easier to perform this type of attack on smaller 
DNS servers that may have limited resources to handle a large number of queries.\\
\\
\textbf{TCP flood}\\
This is another type of attack aimed at consuming the server resources. These latter are consumed by the attacker sending lots of TCP 
connection requests without closing them. The server is forced to allocate resources to handle the TCP connections and, when the 
numbers get larger it may run out of resources.\\
\\
\textbf{DNS water torture}\\
This is a type of indirect attack aimed at consuming the resources of a target authoritative DNS server. The attacker sends a huge number
of queries about properly constructed hostnames. These are made up by two parts: the domain whose authoritative server is the target, and
a random string such that the FQDN cannot exist. That way the recursive NS starts its search until it reaches the authoritative server. This latter
notices the non-existence of the FQDN and sends NXDOMAIN response. All the queries travel until the authoritative server, which is forced to
handle all of them until it is overwhelmed.\\
\\
\textbf{DNS reflection and amplification}\\
This is an indirect attack aimed at consuming the target network bandwidth. It start with the spoofing of the target's IP address, then lots 
of queries are sent to the DNS server using as the source address the spoofed IP address. This latter will receives the responses from the 
DNS server (reflection). The queries are structured in a way that the responses are larger in size (amplification). That way the target's
bandwidth is fulfilled, with the attacker having used just a small quantity of its resources. The effectiveness of this attack is given by 
the amplification factor (AF), measured as the ratio between the size of the response and the size of the query. An easy but effective choice
for the attacker is to perform a type ANY which provides a large AF.\\
\\
Since according to the already cited report \cite{DDoSthreatreport} the most common type of attack is the DNS reflection and amplification,
in this project we focus on this type of attack.




    %\input{./Sections/Mitigation measures_gitignore.tex}
    %Davide
    \section{Experimental setup}

To ensure the success of the project, it is essential to establish a clear methodology and employ a well-defined set of tools.
This section aims to provide a detailed explanation of the methodologies utilized to accomplish the project's objectives.\\
The selected methodologies encompass a systematic approach that allows for the accurate replication of a
DDoS attack while maintaining ethical considerations and minimizing the potential impact on live networks.
These methodologies were carefully chosen to ensure the reliability and validity of the experimental results.
The methodological approach used is the following:\\
\\
\textbf{Why}\\
The objective of this study is to assess the impact of a DoS attack, exploiting the DNS protocol and monitor the reachability
of the targeted device and other network nodes.
By simulating realistic attack scenarios, this study aims to understand the vulnerabilities within the DNS protocol, evaluate network resilience,
and identify potential countermeasures.\\
\\
\textbf{Which/Who}\\
The chosen target for the DDoS attack is a laptop (HP ENVY x360), which acts as a host for a DNS server running BIND9.
Two laptops, specifically a MacBook Pro 14 and another device (to be specified)!!!!!!!!!!!!!!!!, are utilized as vantage points to initiate the attack.\\
\\
\textbf{What}\\
The chosen metrics include measuring the response time for each ICMP or DNS message sent, accounting for potential timeouts.
Furthermore, the CPU and memory utilization of the DNS server were also monitored during the simulation. \\
\\
\textbf{Where}\\
The vantage point is a MacBook Air inside the
network and passive to the attack.\\
To ensure the security and stability of public networks and devices,
the DDoS attack simulations were conducted within a local area network (LAN) environment.

    \section{Tools used}
\textbf{Ping} \\
Ping is a network utility tool used to test the connectivity between two networked devices.
It sends a small packet of data to a specific IP address or hostname and measures the time it takes for that packet to be received and returned.
The result shows the round trip time (RTT), as well as the number of packets sent and received, and any packet loss that may have occurred.\\
While it is a useful tool for testing network connectivity, it has some limitations,
it uses the ICMP protocol to send and receive packets, which may not always be allowed by network firewalls or routers, and it does not support different protocols like TCP or UDP.
This means that if a network is configured to block ICMP traffic, the ping command may not work.
Moreover ping provides only basic information about network connectivity, it does not provide information about bandwidth or the structure of the network.\\
\\
\textbf{Dig} \\
The dig command (short for "domain information groper") is a popular network administration tool used to perform DNS queries.
It allows users to perform DNS lookups to check DNS records and obtain information about DNS configurations.
It is a versatile tool that allows users to specify different query types, such as A, AAAA, MX, TXT, and others.
It can also be used to perform reverse DNS queries, where an IP address is used to retrieve the corresponding domain name.
In addition to provide detailed DNS query results, the dig command can also be used to troubleshoot DNS issues, such as misconfigured DNS servers,
slow DNS resolution times, or DNS cache issues.\\
\\
\textbf{Top} \\
The top command provides real-time monitoring of system resources, such as CPU usage, memory utilization, running processes, and more.
When executed, the top command displays an interactive, dynamic table that continually updates, allowing users to view
the current state of their system and identify any processes consuming excessive resources.
The table is typically sorted by CPU usage by default, but users can customize the sorting order based on their preferences.
The top command also provides options to manipulate the displayed information and perform actions on running processes,
such as terminating or changing their priority.\\
\\
\textbf{Wireshark}\\
Wireshark is a open-source network protocol analyzer.
It is designed to capture, analyze, and display network traffic in real-time.
Wireshark allows users to inspect and interpret the data packets transmitted over a network, providing detailed information
about the communication between different devices.\\
In addition to passive packet capturing, Wireshark offers powerful filtering and search capabilities,
allowing users to focus on specific types of traffic or specific packets of interest.
It also provides features for advanced analysis, such as the ability to reconstruct and view streams of data,
perform statistical analysis, and even export captured data for further investigation or reporting
    \section{DNS Server}
For the simulation of the DDoS attack, a dedicated DNS server was established on a virtual machine running the Ubuntu 24.04 LTS operating system.
The virtual machine was allocated 2GB of RAM and 2 CPU cores to ensure sufficient resources for handling the simulated traffic.\\
To create the DNS server, the open-source software BIND9 was employed. This widely adopted DNS server software offers robust functionality and configurability.
The server was specifically configured to act as the authoritative server for the domain name "ediproject.com."
By assuming authority over this domain, the DNS server can respond to DNS queries and provide legitimate DNS responses during the simulation.\\
To mimic the characteristics of a genuine DNS server, a total of seven NS records and five MX-type records were meticulously added to the DNS server's configuration.
The inclusion of these records introduces diverse amplification factors, enabling the analysis of the DDoS attack's effectiveness based on these factors.
By incorporating varying NS and MX records, the simulation can reflect the behavior of a real DNS server and offer insights into the impact of
different amplification configurations on the severity of the attack.\\



    %\input{./Sections/3_section.tex}
    %\input{./Sections/4_section.tex}
    %\input{./Sections/5_conclusion.tex}
    \newpage
    \printbibliography
\end{multicols}
\end{document}
